\documentclass[11pt]{article}

\usepackage{exscale}
\usepackage{graphicx}
\usepackage{amsmath}
\usepackage{latexsym}
\usepackage{times,mathptm}
\usepackage{epsfig}

\textwidth 6.5truein          
\textheight 9.0truein
\oddsidemargin 0.0in
\topmargin -0.6in

\parindent 0pt          
\parskip 5pt
\def\baselinestretch{1.1}

\begin{document}

\begin{LARGE}
\centerline {\bf CSci 423 Homework 7}
\end{LARGE}
\vskip 0.25cm

\centerline{Due: 12:30 pm, Thursday, 11/7/2019}
\centerline{Daniel Quiroga}

Collaborators: Ethan Young, Will Elliot, Yang Zhang

\begin{enumerate}

\item (2, 2 points) True or false? No explanation needed.
\begin{enumerate}
\item A language is Turing-recognizable if and only if some nondeterministic Turing machine recognizes it.\newline
This is True
\item A language is Turing-decidable if and only if some nondeterministic Turing machine decides it.\newline
This is True
\end{enumerate}

\item (2, 2, 4 points)  Let $\Sigma=\{0,1\}$.
For each of the following $\delta$ functions, describe the corresponding language $L(M)$.
\begin{enumerate}
\item $\delta(q_0,0)=(q_0,B,R)$, $\delta(q_0,1)=(q_1,B,R)$, $\delta(q_1,1)=(q_1,B,R)$,
and $\delta(q_1,B)=(q_{accept},B,R)$. \newline 
$L(M) = 0^n 1^m\ n\ge 0 \  m \ge 1$ 

\item $\delta(q_0,0)=(q_1,1,R)$, $\delta(q_1,1)=(q_2,0,L)$,, $\delta(q_2,1)=(q_0,1,R)$,
and $\delta(q_1,B)=(q_{accept},B,R)$.\newline 
$L(M) =0 \ 1^n \ \ n\ge 0$

\item $\delta(q_0,0)=\{(q_0,1,R),(q_1,1,R)\}$, $\delta(q_1,1)=\{(q_2,0,L)\}$, 
$\delta(q_2,1)=\{(q_0,1,R)\}$, and $\delta(q_1,B)$ $=\{(q_{accept},B,R)\}$. 
(Note: This is a nondeterministic TM.) \newline 
$L(M) = 0(0\cup 1)^*$


\end{enumerate}

\item (8 points) Give the implementation-level description of a Turing machine that {\bf decides} the following language 
$$L=\{w\in\{0, 1\}^*\ |\ w{\rm\ contains\ twice\ as\ many\ 0s\ as\ 1s}\}$$\newline 
TM $=$ on input = w = twice as many 0's as 1's \newline 
step 1: If sees a B, accept \newline 
step 2: move right until sees a 1 or B \newline 
if B, move left -- if sees a 0, reject else: accept \newline 
elif sees 1:  
\begin{enumerate}
\item mark with an X 
\item move left until at the front
\item move right until 0, mark with Y
\item move right until next 0, mark with Y
\end{enumerate}
else (when there is only 0s in the string): reject (might be already included in the first if but just in case) \newline
step 3: go back to the beginning of the string, move right until X \newline goto step 2

\item (5, 5 points) Prove the following closure properties of TDLs.
\begin{enumerate}
\item If $L_1$ and $L_2$ are Turing-decidable, so is $L_1L_2$.
\newline For decidable languages $L_1$ and $L_2$, let $M_1$ and $M_2$ decide them, respectively. 
Design M that decides $L_1L_2$\newline 
TM M $=$ on input w \newline
For every way to split $w = w_1 w_2$ \newline 
$\rightarrow$ Run $M_1$ on $w_1$ & $M_2$ on $w_2$ \newline 
$\rightarrow$ If both accpet, accept \newline
$\rightarrow$ else: continue with the next iteration of the $w_1 w_2$ combo\newline 
If none are accepted after loop, reject \newline \newline 
Since there is a turing machine that describes $L_1 L_2$  , then $L_1 L_2$  is turing-decidable


\item If $L_1$ and $L_2$ are Turing-decidable, so is $L_1\cap L_2$\newline 
For decidable languages $L_1$ and $L_2$, let $M_1$ and $M_2$ decide them, respectively. 
Design M that decides $L_1\cap L_2$ \newline 
TM M $=$ on input w \newline \newline  
$\ \ $ run $M_1$ on w \newline 
$\ \ $ If $M_1$ accepts:\newline 
$\rightarrow$ Run $M_2$ on w \newline 
$\rightarrow$ If $M_2$ accepts, accept w \newline 
$\rightarrow$ else: reject w\newline 
$\ \ $ else: reject w \newline \newline 
Since there is a turing machine that describes $L_1\cap L_2$ , then $L_1\cap L_2$  is turing-decidable
\end{enumerate}

\end{enumerate}

\end{document}