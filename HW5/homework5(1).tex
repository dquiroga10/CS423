\documentclass[11pt]{article}

\usepackage{exscale}
\usepackage{graphicx}
\usepackage{amsmath}
\usepackage{latexsym}
\usepackage{times,mathptm}
\usepackage{epsfig}

\textwidth 6.5truein          
\textheight 9.0truein
\oddsidemargin 0.0in
\topmargin -0.6in

\parindent 0pt          
\parskip 5pt
\def\baselinestretch{1.1}

\begin{document}

\begin{LARGE}
\centerline {\bf CSci 423 Homework 5}
\end{LARGE}
\vskip 0.25cm

\centerline{Due: 12:30 pm, Thursday, 10/17/19}
\centerline{Daniel Quiroga}

Collaborators: Will Elliot, Ethan Young, Yang Zhang

\begin{enumerate}
\item (2, 2, 2, 2 points) Let $L=(a^nb^n\ |\ n\ge0\}$.
\begin{enumerate}
\item Show that $L$ is context-free. \newline 
$S \rightarrow aSb | \epsilon$
\item Show that $L^2$ is context-free.\newline 
$S \rightarrow AA\newline 
A \rightarrow aAb | \epsilon $
\item Show that $L^k$ is context-free for any fixed $k>1$. \newline 
$S \rightarrow AA...A_k\newline 
A \rightarrow a A b| \epsilon $
\item Show that $L^*$ is context-free.\newline 
$S \rightarrow  SS |S| A \newline 
A \rightarrow aAb | \epsilon$\newline 

*just a quick question, but would i be able to simplify the first line to $S \rightarrow SS|A$ *(thanks!)
\end{enumerate}

\item (3 points)  Give a simple description for the language generated by the following grammar $G$.
\begin{eqnarray*}
S&\rightarrow& aSb\ |\ bY\ | \ Ya\\
Y&\rightarrow& bY\ |\ aY\ |\ \epsilon
\end{eqnarray*}
Solution:\newline 
($k\ge1$)$a^k (b (a\cup b)^* \cup (a\cup b)^*a)b^k\  \cup \ b(a\cup b)^*\  \cup \ (a\cup b)^*a$

\item (3, 4, 3 points) Consider the language $F=\{a^ib^jc^k\ |\ i, j, k\ge0 {\rm\ and\ if\ }i=1{\rm\ then\ }j=k\}$.
\begin{enumerate}
\item Define $F$ as a union of three languages, depending of the value/range of $i$, i.e., $i=0$, $i=1$, and $i\ge2$. \newline 
$F = \{a^ib^jc^k\ | i = 0, j,k \ge0\}\  \cup\ \{a^ib^nc^n\ | i =1, n \ge0 (n = j = k)\} \ \cup \ \{a^ib^jc^k\ | i\ge2, j,k \ge0 \}$
\item Prove that $F$ is not regular by closure properties.\newline 
$F_1 = \{ a^i b^j c^k | i,j,k \ge0 \}  \newline 
F_2 = \{ a^i b^j c^k | i = 1, j = k \ge0\} \newline 
we \ assume\ that \ F_1 \ and \ F_2 \ are\ both \ regular \ languages \ therefore \ the \ intersection \ must\ be \ regular. \newline 
F = F_1 \cap F_2 = \{ a^1 b^n c^n | n \ge0 \ \ (n = j = k) \}  \newline$
we have proved that $ b^n c^n$ is not regular therefore we have a contradiction. F is not regular. 
\item Construct a context-free grammar for $F$, based on your answer to (a).\newline 
$S \rightarrow S_1 | S_2 | S_3 \newline \newline 
S_1 \rightarrow bS_1 | S_1c |\epsilon \newline \newline 
S_2 \rightarrow aA \newline 
A \rightarrow bAc | \epsilon \newline \newline 
S_3 \rightarrow aaB \newline
B \rightarrow aB | C \newline
C \rightarrow bC | Cc | \epsilon
$
\end{enumerate}

\item (3, 3, 3 points) Give a simple CFG for each of the following languages, where $i, j, k\ge0$.

\begin{enumerate}
\item $A=\{a^ib^jc^k\in \{a, b, c\}^*\ |\ i\not=j {\rm\ or\ }j\not=k\}$. \newline 
$
S \rightarrow S_1 | S_2 \newline \newline 
S_1 \rightarrow A_1|B_1 \ (the\  case\  where\  i\  is\  not\  equal\  to\  j) \newline 
A_1 \rightarrow aA_1b | C_1E_1 \newline 
C_1 \rightarrow aC_1 | a \newline
B_1 \rightarrow aB_1b|D_1E_1 \newline
D_1 \rightarrow bD_1 | b \newline 
E_1 \rightarrow cE_1 | \epsilon \newline \newline 
S_2 \rightarrow E_2A_2 | E_2B_2\  (the\  case\  where\  j\  is\  not\  equal\  to\  k) \newline 
A_2 \rightarrow bA_2c | C_2 \newline 
C_2 \rightarrow bC_2 | b \newline
B_2 \rightarrow bB_2c|D_2 \newline
D_2 \rightarrow cD_2 | c \newline 
E_2 \rightarrow aE_2 | \epsilon
$
\item $B=\{a^ib^jc^k\in \{a, b,, c\}^*\ |\ |i-j|=k\}$ \newline 
reword this into two situations: $i =k + j  \ \ \ \ j = k+i$ \newline 
$S \rightarrow S_1 | S_2 \newline \newline 
S_1 \rightarrow aS_1c | A \newline 
A \rightarrow aAb | \epsilon \newline \newline 
S_2 \rightarrow AB \newline 
A \rightarrow aAb | \epsilon \newline 
B \rightarrow bBc |\epsilon
$
\item $C=\{a^ib^jc^k\in\{a, b, c\}^*\ |\ i+j=k\}$. \newline 
$S \rightarrow aSc | A \newline 
A \rightarrow bAc | \epsilon
$
\end{enumerate}


\end{enumerate}

\end{document}