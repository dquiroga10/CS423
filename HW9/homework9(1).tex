\documentclass[11pt]{article}

\usepackage{exscale}
\usepackage{graphicx}
\usepackage{amsmath}
\usepackage{latexsym}
\usepackage{times,mathptm}
\usepackage{epsfig}

\textwidth 6.5truein          
\textheight 9.0truein
\oddsidemargin 0.0in
\topmargin -0.6in

\parindent 0pt          
\parskip 5pt
\def\baselinestretch{1.1}

\begin{document}

\begin{LARGE}
\centerline {\bf CSci 423 Homework 9}
\end{LARGE}
\vskip 0.25cm

\centerline {Due:12:30 pm, Tuesday, 11/26}
\centerline{Daniel Quiroga}

Collaborators:

\begin{enumerate}

\item (1, 1, 1, 3 points) In class, we learned that $A_D$ is non-TR, $A_{TM}$ and $HALT_{TM}$ are TR but non-TD. What can you say about their complements? Circle the correct answers below.
\begin{enumerate}
\item ${\overline A_D}$ is \qquad (i) TD; \qquad (ii) TR but non-TD; \qquad (iii) non-TR. ANSWER: ii
\item ${\overline A_{TM}}$ is \qquad (i) TD; \qquad (ii) TR but non-TD; \qquad (iii) non-TR. ANSWER: iii
\item ${\overline {HALT}_{TM}}$ is \qquad (i) TD; \qquad (ii) TR but non-TD; \qquad (iii) non-TR. ANSWER: iii
\end{enumerate}

In addition, justify your answer to (a) by giving a proof.


\item (6 points) Prove that $ES_{TM}=\{<M>\ |\ M{\rm\ accepts\ }\epsilon\}$ is non-TD. (Hint: Reduce from $A_{TM}$.) \newline 
Assume $ES_{TM}$ is TD. Then there is a Turing Machine R that decides $ES_{TM}$. \newline So TM R given a Turing Machine M it will accept if $\epsilon \in L(M)$ or reject otherwise. We will try to define a Turing Machine S that decides $A_{TM}$ (reduction sign)  with this information. \newline 
Define TM M' = on input x $\rightarrow$ Run M on w. \newline 
$L(M') = \sum^* if\  w \in L(M) | \  \emptyset\ otherwise $ \newline 
M' accpets $\epsilon$ iff $w \in L(M)$ \newline Then we have a turning machine S that would decide  $A_{TM} \rightarrow$ CONTRADICTION! \newline 
$ES_{TM}$ is Turing-undecidable. 

\item (6 points) Let $T=\{<M>\ |\ M{\rm\ is\ a\ TM\ that\ accepts\ }w^R{\rm\ whenever\ it\ accepts\ }w\}$. 
Recall that $w^R$ is the reverse of $w$. Prove that $T$ is non-TD. 

Hints: (1) Reduce from $A_{TM}$. (2) For any $M$ and $w$, can you define a TM $M_1$ such that $L(M_1)=\{01, 10\}$ if $M$ accepts $w$ and $L(M_1)=\{01\}$ if $M$ does not accept $w$?



\item (6 points) Prove that it is undecidable whether $L(M_1)\subseteq L(M_2)$ for any given TMs $M_1$ and $M_2$. (Hint: Reduce from $EQ_{TM}$.)\newline 
In order to prove equality or inequality using subsets we need two principles to hold: for equality we need  $L(M_1)\subseteq L(M_2)$ and  $L(M_2)\subseteq L(M_1)$ to both come out true. If one of them fails than we have inequality. \newline 
We first assume that the subset problem is TD. Then there must be a turning machine that decides it. \newline 
We design a Turning Machine E that will accept the input $<M_1, M_2>$: accept if the first input is a subset of the second, reject otherwise. We will try to define a Turning machine S that decides $EQ_{TM}$. Since this subset is a smaller part of the entire equality theorem. \newline 
Define E = on input x $\rightarrow$ run input $<M_1, M_2>$ if it accepts, then run $<M_2, M_1>$ if it accepts then have TM S accept, if any steps rejects have TM S reject. E accpets on both of the inputs iff $L(M_1) = L(M_2)$ iff S accepts $<M_1, M_2>$. \newline 
We have designed a TM S that decides $EQ_{TM} \rightarrow$ CONTRADICTION!  \newline 
The problem is Turing-undecidable

\item (6 points) Prove that the question ``Does $L(M)$ contain any string of length $5$'' is undecidable. (Hint: Reduce from $A_{TM}$.)
We assume that the question is decidable. Then there must be TM R that decides the question So TM R give another TM M will accept if the length of the string is exactly 5, reject otherwise. We will try to define TM S that decides $A_{TM}$ with this information. \newline 
Define TM M' = on input x $\rightarrow$ Run M on w. \newline 
$L(M') = \sum^* if\  w \in L(M) | \  \emptyset\ otherwise $ \newline 
M' accepts $w of length 5$ iff $w \in L(M)$ \newline 
Then this would mean that TM S would be able to decide $A_{TM} \rightarrow$ CONTRADICITION! \newline 
The question is undecidable. 

\end{enumerate}

\end{document}